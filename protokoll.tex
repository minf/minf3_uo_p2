\documentclass[a4paper,10pt]{article}

\usepackage{geometry}

\usepackage[utf8x]{inputenc}
\usepackage[bookmarks,colorlinks=false,pdfborder={0 0 0}]{hyperref}
\hypersetup{pdftitle={Unternehmensorientierung - Geschäftsidee: gamergeld.de}}
\usepackage{url}
\usepackage[ngerman]{babel}
\usepackage{graphicx}
\usepackage{listings}

\parindent 0pt
\parskip 10pt

\title{Unternehmensorientierung - Geschäftsidee: gamergeld.de}
\author{Erik Andresen \and Andreas Basener \and Jan Depke \and Andreas Krohn \and Benjamin Vetter}

\begin{document}

\maketitle

\section{Geschäftsidee}
\emph{Zentralisierte Zahlungsabwicklung für Browsergames/Free-to-Play Games}

Kooperation mit Gameherstellern

Gamehersteller sollten/werden mit uns kooperieren, weil
\begin{itemize}
  \item weniger Implentierungsaufwand (vor allem: neue Spiele)
  \item Verlinkung von unserer Seite - damit: Bekanntheitsgewinn
\end{itemize}

Wir bieten eine API für Gamehersteller an.
Die Wechselkurse zur jeweiligen Gamewährung werden per API übergeben.

Spieler wird z.B. beim Kauf eines Items auf gamergeld.de redirected und zahlt dort den geforderten Betrag bzw. belastet sein Konto.
Zahlung über Paypal, clickandbuy, giropay, Kreditkarte, Bankeinzug, sofortüberweisung, Moneybookers, Bitcoins, Prepaid, ..

\subsection{Risiken}
\begin{itemize}
  \item Zahlungsausfall
  \item Klar als Vermittler und nicht als Anbieter der Items kennzeichnen (Regress..)
\end{itemize}

\section{Tragfähigkeit}\label{labelTragfaehigkeit}

\subsection{Marktuntersuchung}\label{labelMarktuntersuchung}
Depke
\begin{itemize}
  \item Gibt es ähnliche Anbieter?
  \item Wie hoch sind die zu zahlenden Preise bei Browsergames?
  \item Wieviele (zahlende) User in Browsergames?
\end{itemize}

\subsection{Technische Machbarkeit}\label{labelTechMach}

Um unsere Dienstleistung für Gamehersteller anbieten zu können verwenden wir
die individuellen APIs der spezifischen Zahlungsanbieter und bieten
unsererseits eine einheitliche API für die Gamehersteller an. Alle in diesem
Dokument genannten Zahlungsanbieter verfügen über derartige APIs, sodass diese
problemlos von uns integriert werden können. Jeder Gamehersteller kann
Zahlungstransaktionen über unsere abstrahierte API durchführen und dadurch alle
von uns verwendeten Zahlungsanbieter unterstützen. 

\subsubsection{API}
Um die Verwendung unserer API möglichst einfach zu gestalten und die API
unsererseits nicht in allen gängigen Programmiersprachen implementieren zu
müssen, bietet sich die Verwendung gängiger Protokolle an, um die API
programmiersprachen- und plattformunabhängig zu gestalten. Die API werden wir
daher REST-basiert implementieren. Durch den HTTP-Unterbau von REST kann die
API in allen gängigen Programmiersprachen problemlos und ohne zusätzlichen
Aufwand unsererseits verwendet werden. Der grobe Protokollablauf für eine
einzelne Transaktion sieht bspw. wie folgt aus:

\begin{enumerate}
  \item Der Gamehersteller erstellt eine neue Transaktion für das zuvor im Account
    erstellte Game und über einen bestimmten Betrag
  \item Die Response teilt die Transaktionskennung mit
  \item Der Gamer wird zu gamergeld.de weitergeleitet oder alternativ wird gamergeld.de
    innerhalb eines iFrames angezeigt um die Transaktion innerhalb des Spiels durchzuführen
  \item Auswahl der Zahlungsart und ggf. des Zahlungsanbieters durch den Gamer, ggf. Login
  \item Durchführung der Zahlung
  \item Benachrichtigung des Zahlungsanbieters über Erfolg/Misserfolg der Transaktion,
    bspw. über eine vom Hersteller spezifizierte URL und mit Transaktionskennung
  \item Damit ist das Protokoll abgeschlossen
\end{enumerate}

\subsubsection{Infrastruktur}
Um auch auf Wachstum über die in diesem Dokument spezifizierten Erwartungen
hinaus eingestellt zu sein und eine grundsätzlich leicht skalierbare
Architektur bei planbaren Kosten zu realisieren, bietet sich die Verwendung von
Cloud-Angeboten an. Indes muss die Verfügbarkeit der verwendenten Infrastruktur
überdurchschnittlich sein, da eine Nicht-Verfügbarkeit von gamergeld.de einen
Zahlungsausfall seitens unserer Kunden nach sich zieht.

Dabei kommen Cloud-Lösungen bzgl. der verwendeten Server-, Storage- und
Datenbank-Infrastruktur, ebenso wie Load-Balancer-Lösungen in Frage.
Storage-Lösungen sind dabei jedoch von untergeordnetem Interesse, da diese nur
bzgl. Content-Delivery (CDN) für die statischen Mediendaten, die in die Website
eingebettet sind, in Frage kommen. Große und umfangreiche Mediendaten werden
hingegen von gamergeld.de nicht vorgehalten. Beispiele für derartige
Cloud-Lösungen sind bspw. die Angebote von Amazon \cite{Amazon} und Rackspace
\cite{Rackspace}. Die Kosten eines durchschnittlichen Cloud-Servers (EC2) von
Amazon belaufen sich derzeit bei rund \$227.50 jährlich \cite{Amazon}. Amazon
gibt die Verfügbarkeit ihrer Services mit 99.95\% an \cite{Amazon}. Dennoch
müssen, zwecks Redundanz und Fehlertoleranz, mehrere Server in vorzugsweise
mehreren Rechenzentren vorgehalten werden, sodass die Kosten dafür zu Beginn
bei \$800-1500 jährlich liegen. Die exakte Anzahl zu verwendenden Servern ist
abhängig vom Traffic und Transaktionsvolumen. Unsere Architektur ist auch
kurzfristig skalierbar.

Bzgl. der Datenbanken verwenden wir keine zusätzlichen Cloud-basierten
Datenbanken wie bspw. Amazon RDS, sondern verwenden MySQL- und
MongoDB-Datenbanken, die sich auf den von uns betriebenen Servern befinden.
Das hat den zusätzlichen Vorteil, dass bspw. Kundendaten ausschließlich auf
Systemen gespeichert sind, die von uns administriert werden. Die
MySQL-Datenbanken werden für Account- und andere Kundendaten verwendet,
betreiben eine Master-Slave und/oder Master-Master Replikation und sind daher
bzgl. der Read-Anweisungen gut skalierbar. Bzgl. der Write-Anweisungen wird
jedoch nur moderate Skalierbarkeit erreicht. Häufige Änderungen an Account- und
Kundendaten sind jedoch idR. nicht zu erwarten, sodass hierbei keine Probleme
entstehen. Für die Speicherung von Transkationsdaten sind die MySQL-Systeme bei
hohen Transaktionsvolumina jedoch, ohne umfangreiche Maßnahmen, nur bedingt
geeignet. Daher vewenden wir für die Transaktionensdaten MongoDB-Datenbanken,
ebenfalls mit Master-Slave und Master-Master Replikation, da NoSQL-Datenbanken
bzgl. der Write-Anweisungen einfacher zu skalieren sind\footnote{Bspw. da
Sharding bei denormalisierten Datenbanken deutlich einfacher zu realisieren
ist.}.

\subsubsection{Sicherheit}
Insider-Angriffe seitens des Cloud-Anbieters können wir nur z.T.
berücksichtigen. Jedoch dürfen Account-, Kunden-, Kreditkarten- und
Transaktionsdaten serverseitig ausschließlich auf Systemen gespeichert werden,
die von uns administriert werden. Da wir keine zusätzlichen Cloud-basierten
Datenbanken verwenden ist dies jedoch gewährleistet.

Da gamergeld.de u.a. Kreditkartendaten entgegennimmt und speichert, gelten für
die verwendeten System besonders hohe Sicherheitsanforderungen und
Schutzkategorien. Unter anderem müssen unsere Systeme die PCI-Compliance Tests
bestehen \cite{PCI}, die für alle Unternehmen und Systeme, die
Kreditkartendaten verarbeiten, gelten. Das ist mit entsprechenden
Schutzmaßnahmen, als auch mit Kosten für die PCI-Compliance-Tests verbunden.
Die Kosten für derartige Tests müssen idR. vierteljährlich durchgeführt werden,
sind abhängig von der Anzahl verwendeter IP-Adressen als auch von den
Transaktionvolumina. Pro IP-Adresse und Quartal belaufen sich die Kosten pro
IP-Adresse daher bei ca. 220 EUR \cite{PCI:Costs}, also ca. 4000 EUR jährlich
für das beschriebene Setup.

Bzgl. der Serversicherheit werden ausserdem typische Best-Practices, wie
Policies und Patch-Zyklen, Firewalls, verschlüsselte Kommunikation (SSH, SSL),
Authentifikation, etc. verwendet.


\newpage
\subsection{Finanzierung und wirtschaftliche Machbarkeit}\label{labelWirtMach}
%Gesamtplan 3-5 Jahre

Für die Berechnung unserer Finanzierung und der wirtschaftlichen Machbarkeit werden folgende Annahmen getroffen:

Als einzelne Kostenpositionen erwarten wir folgende Posten:
\begin{itemize}
\item Fixkosten
 \begin{itemize}
\item Serverkosten
\item Lizenzkosten
\item Personalkosten
\item Büromiete
 \end{itemize}
\item variable Kosten
\begin{itemize}
\item Marketingkosten
\item Entwicklungskosten
\item Bürobedarf
\end{itemize}
\end{itemize}

Die Kosten für die Server ergeben sich aus den Betrachtungen im Kapitel \ref{labelTechMach}. Die vermuteten Kosten für das Büro sind dem aktuellen Hamburger Mietspiegel entnommen. Die Personalkosten verteilen sich auf die fünf Mitglieder dieses Unternehmens.

Wie in Kapitel \ref{labelMarktuntersuchung} beschrieben erwarten wir folgende Kunden um Umsätze zu erlösen:
\begin{itemize}
\item Kunde A
\item Kunde B
\item Kunde C
\item Kunde D
\item Kunde E
\end{itemize}

Wir wollen mit unsren Kunden eine langfristige Geschäftsbeziehung eingehen. Wir setzen dabei darauf, dass wir nur einige wenige Kunden benötigen. Durch diese wenigen Kunden können wir aber hohe Umsätze erzielen.
Dabei wollen wir im ersten Quartal 2012 mit Kunde A anfangen und gemeinsam mit dem Kunden die Integration der Zahlungsabwicklung erweitern. Dadurch erhöhen sich die Umsätze in den darauffolgenden Quartalen steig.\\
Ab dem dritten Quartal 2012 gewinnen wir Kunde B und Kunde C als neue Kunden hinzu und im ersten Quartal 2013 zusätzlich Kunde D.\\
Ab dem ersten Quartal 2014 wollen wir in den internationalen Markt expandieren. Dazu wollen wir Kunde E gewinnen.

\subsubsection*{Gewinn und Verlustrechnung}
%24 auf monatlicher Basis, danach jährlich
\begin{figure}[htbp]
	\centering
	\includegraphics[width=1\textwidth]{geschaeftsplan/GuV.png} 
	\caption{GuV}
	\label{picGuV}
\end{figure}
Für die Berechnung des Gewinns müssen wir berechnen, wie viel ein Spieler im Durchschnitt pro Transaktion ausgibt. Dieser Betrag liegt nach unseren Recherchen bei XX,XX Euro. Dardurch entstehen uns Kosten für die Durchführung der Transaktion. Dieser Betrag liegt im Schnitt bei YY,YY Euro. Diese Kosten, zuzüglich eines Gewinnaufschlags von ZZ,ZZ\% behalten wir für uns ein, der Rest wird an den Kunden weiter gegeben.\\
Dieser Gewinnaufschlag bestimmt unsere Umsätze.

Die erwarteten und aufaddierten Umsätze und Ausgaben sind in der Grafik \ref{picGuV} pro Quartal für die Jahre 2012 bis 2014 dargestellt.

\subsubsection*{Liquiditätsrechnung}
%24 auf monatlicher Basis, danach jährlich
\begin{figure}[htbp]
	\centering
	\includegraphics[width=1\textwidth]{geschaeftsplan/GuVkummuliert.png}
	\caption{GuV kumuliert}
	\label{picGuVkum}
\end{figure}

Der Break-Even Punkt wird im ersten Quartal 2013 überschritten. Bis zum vorhergehenden Quartal dahin addieren sich die Verluste auf ca. 90.000,- Euro (s. Grafik \ref{picGuVkum}). Dieser Betrag wird mindestens benötigt, um die zu erwartenden Kosten zu decken.\\
Die Deckung der Kosten wird durch eine Bankfinanzierung geschehen. Dazu nehmen wir einen Kredit in Höhe von 100.000,- Euro auf. Laufzeit 4 Jahre mit einem Zinssatz in Höhe von 5\%. Die monatlichen Tilgungsraten betragen 2.777,78 Euro. Die Kosten sind in den übrigen Berechnungen bereits enthalten.


%\begin{itemize}
%  \item Unsere Finanzierung (Prozentsatz der jeweiligen Transaktionen, Pauschale, Vertrag mit Spielehersteller...)
%  \item Wieviele Nutzer brauchen wir bei welchem Umsatz?
%\end{itemize}



\newpage
\section{Optimierung durch Analyse der Informationen}\label{labelRelDaten}
% Von Anfang an alle relevanten Daten zu sammeln und in geeigneter Form festzuhalten (Stich-
% worte: Metriken, Datawarehouse, Business Intelligence)


Zur ständigen Qualitätsverbesserung und Entwicklung des Vorsprungs gegenüber Wettbewerbern müssen unsere Dienste ständig optimiert werden.
Dazu werden wir über unsere Webseite große Mengen an Daten erheben und daraus folgende Informationen erschließen:\\
Wie oft bzw wann benutzen die Spieler unsere Seite, bzw die Spieleseite zur Zahlungsabwicklung?\\
Wann wird ein Zahlungsvorgang abgebrochen? Gibts es korrelierende Daten, die man den Browserspielen, der Zahlungsart oder einer Postleitzahl/Ort zuorden kann?\\
Kundenflüsse beobachten: Gibt es Spieler, die von uns auf andere Spiele aufmerksam werden? Welche ähnlichen Spiele kann man Zahlungswilligen Spielern noch empfehlen?\\
Kann man größere Mengen von Zahlungsausfällen zu bestimmten Regionen/Postleitzahlgebieten zuorden? Oder gibt es Browserspiele mit einer besonders hohen Quote an Zahlungsausfällen?
Wenn sich solche Browserspiele identifizieren lassen kann man diese mit höhren Gebühren belasten als Browserspiele, die höhere Umsätze erzielen bei geringerer Ausfallquote.
Bei Spielern aus Regionen mit hoher Zahlungsausfallquote kann man die Zahlungsart auf weniger Risikoreiche varianten wie Vorauskasse beschränken.\\
Erkennung rückläufiger Umsätze: Gibt es Browserspiele oder Zahlungsarten die insgesamt profitabler sind als andere? Welche Browserspiele erzielen den meisten Gewinn?
Tendenzielle Rückläufe bei den Zahlungen müssen schnell erfasst und gegebenfalls den Spielemachern mitgeteilt werden damit diese rechtzeitig neue Innovationen (und damit Gründe für Spieler,
Geld auszugeben) in ihr Spiel einbauen können.

Gesammelt werden die Daten ausschließlich über das Benutzerverhalten.
In den Daten muss genau gespeichert werden welcher Spieler wann vom welchem Spiel welchen Zahlungsvorgang bei welchem Anbieter mit welcher Summe durchführt oder wo abbricht:
Zur Auswertung der Daten werden multidimensional fähige Werkzeuge benötigt, also Werkzeuge aus dem Bereich des "`Online Analytical Processing"' um die für uns wichtigen Ergebnisse
nach z.B. Wohnort, Spiel und Zahlungsart darzustellen.\\
Eine mögliche Information kann auch sein, daß Wartungsarbeiten am besten Nachts zwischen 3 und 4 durchgeführt werden können, da dort die Wahrscheinlichkeit eines Zahlungausfalls am geringsten ist.\\
Informationen über Abbrüche beim Zahlungsvorgang und die Beziehungen welcher Spieler von welchem Portalen unsere Dienste benutzen kann man eher mit Web-Mining gewinnen.
Dabei können so möglicherweise keine Information gewonnen werden, wieso der Zahlungsvorgang abgebrochen wurde: War der User verunsichert oder hat er die Anleitung nicht verstanden?
Um diese Fragen zu beantworten sind weitere Usability-Studien erforderlich.\\
Auch Usability-Verbesserungen sind möglich, wie z.B. Zahlungsarten die je nach Spiel/User häufiger verwendet werden, können auf der Seite weiter oben angesiedelt werden.

\section{Projektplan/Geschäftsplan}\label{labelGeschaeftsplan}
Krohn
% Ergebnisse der Validierung zusammenfassen...

\bibliographystyle{plain}
\bibliography{protokoll}

\end{document}

% vim: fileencoding=utf8
