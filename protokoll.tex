\documentclass[a4paper,10pt]{article}

\usepackage{geometry}

\usepackage[utf8x]{inputenc}
\usepackage[bookmarks,colorlinks=false,pdfborder={0 0 0}]{hyperref}
\hypersetup{pdftitle={Unternehmensorientierung - Geschäftsidee: gamergeld.de}}
\usepackage{url}
\usepackage[ngerman]{babel}
\usepackage{graphicx}
\usepackage{listings}

\parindent 0pt
\parskip 10pt

\title{Unternehmensorientierung - Geschäftsidee: gamergeld.de}
\author{Erik Andresen \and Andreas Basener \and Jan Depke \and Andreas Krohn \and Benjamin Vetter}

\begin{document}

\maketitle

\section{Geschäftsidee}
\emph{Zentralisierte Zahlungsabwicklung für Browsergames/Free-to-Play Games}

Kooperation mit Gameherstellern

Gamehersteller sollten/werden mit uns kooperieren, weil
\begin{itemize}
  \item weniger Implentierungsaufwand (vor allem: neue Spiele)
  \item Verlinkung von unserer Seite - damit: Bekanntheitsgewinn
\end{itemize}

Wir bieten eine API für Gamehersteller an.
Die Wechselkurse zur jeweiligen Gamewährung werden per API übergeben.

Spieler wird z.B. beim Kauf eines Items auf gamergeld.de redirected und zahlt dort den geforderten Betrag bzw. belastet sein Konto.
Zahlung über Paypal, clickandbuy, giropay, Kreditkarte, Bankeinzug, sofortüberweisung, Moneybookers, Bitcoins, Prepaid, ..

\subsection{Risiken}
\begin{itemize}
  \item Zahlungsausfall
  \item Klar als Vermittler und nicht als Anbieter der Items kennzeichnen (Regress..)
\end{itemize}

\section{Tragfähigkeit}

\subsection{Marktuntersuchung}
Depke
\begin{itemize}
  \item Gibt es ähnliche Anbieter?
  \item Wie hoch sind die zu zahlenden Preise bei Browsergames?
  \item Wieviele (zahlende) User in Browsergames?
\end{itemize}

\subsection{Technische Machbarkeit}
Vetter
\begin{itemize}
  \item Schnittstellen zu Zahlungsanbietern
  \item Infrastruktur
\end{itemize}

\subsection{Finanzierbarkeit/wirtschaftl. Machbarkeit}
Basener
\begin{itemize}
  \item Unsere Finanzierung (Prozentsatz der jeweiligen Transaktionen, Pauschale, Vertrag mit Spielehersteller...)
  \item Wieviele Nutzer brauchen wir bei welchem Umsatz?
\end{itemize}

\section{Relevante Daten}
Andresen

\section{Projektplan/Geschäftsplan}
Krohn
% Ergebnisse der Validierung zusammenfassen...

\end{document}
