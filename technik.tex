
Um unsere Dienstleistung für Gamehersteller anbieten zu können verwenden wir
die individuellen APIs der spezifischen Zahlungsanbieter und bieten
unsererseits eine einheitliche API für die Gamehersteller an. Alle in diesem
Dokument genannten Zahlungsanbieter verfügen über derartige APIs, sodass diese
problemlos von uns integriert werden können. Jeder Gamehersteller kann
Zahlungstransaktionen über unsere abstrahierte API durchführen und dadurch alle
von uns verwendeten Zahlungsanbieter unterstützen. 

\subsubsection{API}
Um die Verwendung unserer API möglichst einfach zu gestalten und die API
unsererseits nicht in allen gängigen Programmiersprachen implementieren zu
müssen, bietet sich die Verwendung gängiger Protokolle an, um die API
programmiersprachen- und plattformunabhängig zu gestalten. Die API werden wir
daher REST-basiert implementieren. Durch den HTTP-Unterbau von REST kann die
API in allen gängigen Programmiersprachen problemlos und ohne zusätzlichen
Aufwand unsererseits verwendet werden. Der grobe Protokollablauf für eine
einzelne Transaktion sieht bspw. wie folgt aus:

\begin{enumerate}
  \item Der Gamehersteller erstellt eine neue Transaktion für das zuvor im Account
    erstellte Game und über einen bestimmten Betrag
  \item Die Response teilt die Transaktionskennung mit
  \item Der Gamer wird zu gamergeld.de weitergeleitet oder alternativ wird gamergeld.de
    innerhalb eines iFrames angezeigt um die Transaktion innerhalb des Spiels durchzuführen
  \item Auswahl der Zahlungsart und ggf. des Zahlungsanbieters durch den Gamer, ggf. Login
  \item Durchführung der Zahlung
  \item Benachrichtigung des Zahlungsanbieters über Erfolg/Misserfolg der Transaktion,
    bspw. über eine vom Hersteller spezifizierte URL und mit Transaktionskennung
  \item Damit ist das Protokoll abgeschlossen
\end{enumerate}

\subsubsection{Infrastruktur}
Um auch auf Wachstum über die in diesem Dokument spezifizierten Erwartungen
hinaus eingestellt zu sein und eine grundsätzlich leicht skalierbare
Architektur bei planbaren Kosten zu realisieren, bietet sich die Verwendung von
Cloud-Angeboten an. Indes muss die Verfügbarkeit der verwendenten Infrastruktur
überdurchschnittlich sein, da eine Nicht-Verfügbarkeit von gamergeld.de einen
Zahlungsausfall seitens unserer Kunden nach sich zieht.

Dabei kommen Cloud-Lösungen bzgl. der verwendeten Server-, Storage- und
Datenbank-Infrastruktur, ebenso wie Load-Balancer-Lösungen in Frage.
Storage-Lösungen sind dabei jedoch von untergeordnetem Interesse, da diese nur
bzgl. Content-Delivery (CDN) für die statischen Mediendaten, die in die Website
eingebettet sind, in Frage kommen. Große und umfangreiche Mediendaten werden
hingegen von gamergeld.de nicht vorgehalten. Beispiele für derartige
Cloud-Lösungen sind bspw. die Angebote von Amazon \cite{Amazon} und Rackspace
\cite{Rackspace}. Die Kosten eines durchschnittlichen Cloud-Servers (EC2) von
Amazon belaufen sich derzeit bei rund \$227.50 jährlich \cite{Amazon}. Amazon
gibt die Verfügbarkeit ihrer Services mit 99.95\% an \cite{Amazon}. Dennoch
müssen, zwecks Redundanz und Fehlertoleranz, mehrere Server in vorzugsweise
mehreren Rechenzentren vorgehalten werden, sodass die Kosten dafür zu Beginn
bei \$800-1500 jährlich liegen. Die exakte Anzahl zu verwendenden Servern ist
abhängig vom Traffic und Transaktionsvolumen. Unsere Architektur ist auch
kurzfristig skalierbar.

Bzgl. der Datenbanken verwenden wir keine zusätzlichen Cloud-basierten
Datenbanken wie bspw. Amazon RDS, sondern verwenden MySQL- und
MongoDB-Datenbanken, die sich auf den von uns betriebenen Servern befinden.
Das hat den zusätzlichen Vorteil, dass bspw. Kundendaten ausschließlich auf
Systemen gespeichert sind, die von uns administriert werden. Die
MySQL-Datenbanken werden für Account- und andere Kundendaten verwendet,
betreiben eine Master-Slave und/oder Master-Master Replikation und sind daher
bzgl. der Read-Anweisungen gut skalierbar. Bzgl. der Write-Anweisungen wird
jedoch nur moderate Skalierbarkeit erreicht. Häufige Änderungen an Account- und
Kundendaten sind jedoch idR. nicht zu erwarten, sodass hierbei keine Probleme
entstehen. Für die Speicherung von Transkationsdaten sind die MySQL-Systeme bei
hohen Transaktionsvolumina jedoch, ohne umfangreiche Maßnahmen, nur bedingt
geeignet. Daher vewenden wir für die Transaktionensdaten MongoDB-Datenbanken,
ebenfalls mit Master-Slave und Master-Master Replikation, da NoSQL-Datenbanken
bzgl. der Write-Anweisungen einfacher zu skalieren sind\footnote{Bspw. da
Sharding bei denormalisierten Datenbanken deutlich einfacher zu realisieren
ist.}.

\subsubsection{Sicherheit}
Insider-Angriffe seitens des Cloud-Anbieters können wir nur z.T.
berücksichtigen. Jedoch dürfen Account-, Kunden-, Kreditkarten- und
Transaktionsdaten serverseitig ausschließlich auf Systemen gespeichert werden,
die von uns administriert werden. Da wir keine zusätzlichen Cloud-basierten
Datenbanken verwenden ist dies jedoch gewährleistet.

Da gamergeld.de u.a. Kreditkartendaten entgegennimmt und speichert, gelten für
die verwendeten System besonders hohe Sicherheitsanforderungen und
Schutzkategorien. Unter anderem müssen unsere Systeme die PCI-Compliance Tests
bestehen \cite{PCI}, die für alle Unternehmen und Systeme, die
Kreditkartendaten verarbeiten, gelten. Das ist mit entsprechenden
Schutzmaßnahmen, als auch mit Kosten für die PCI-Compliance-Tests verbunden.
Die Kosten für derartige Tests müssen idR. vierteljährlich durchgeführt werden,
sind abhängig von der Anzahl verwendeter IP-Adressen als auch von den
Transaktionvolumina. Pro IP-Adresse und Quartal belaufen sich die Kosten pro
IP-Adresse daher bei ca. 220 EUR \cite{PCI:Costs}, also ca. 4000 EUR jährlich
für das beschriebene Setup.

Bzgl. der Serversicherheit werden ausserdem typische Best-Practices, wie
Policies und Patch-Zyklen, Firewalls, verschlüsselte Kommunikation (SSH, SSL),
Authentifikation, etc. verwendet.

