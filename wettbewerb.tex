\subsection{Wettbewerber}

\begin{itemize}
 \item Zahlungsabwicklung der Finanzinstitute
 \item Zahlungsabwicklung durch Aggregatoren
\end{itemize}

%------------------------------------------------------------------------------------

\subsubsection{Anbietertypen}

\begin{itemize}
 \item Zahlungsabwicklung der Finanzinstitute
 \item Zahlungsabwicklung durch Aggregatoren
\end{itemize}
 
%------------------------------------------------------------------------------------

\subsubsection{Realisierungstypen}

\begin{itemize}
 \item Debit-Card
 \item eigenständige Pseudowährungen
 \item Meta-Währungen (angestrebt)
 \item "man in the middle" für klassische Finanzinstitute
 \item Aggregatoren unterschiedlicher Realisierungstypen
\end{itemize}

%------------------------------------------------------------------------------------

\subsubsection{Angebotskomponenten}

\begin{itemize}
 \item Zahlungsabwicklung
 \item Transaktionsprüfung
 \item Kundenprüfung
 \item Supportdienstleistung
 \item Fraud Protection
\end{itemize}

%------------------------------------------------------------------------------------

\subsubsection{Bekannte Dienstleister}

\begin{itemize}
 \item paysafecard (innerdeutsch relevant)
 \item paypal   (innerdeutsch relevant)
 \item Visa, Mastercard, Amex, Discover (innerdeutsch relevant)
 \item Giropay (innerdeutsch relevant)
 \item Bank Transfer
 \item Wallie Card
 \item Moneybookers
 \item Ultimate Game Card
 \item Ukash
 \item Visa Electron, V.me by Visa (Paypal Klon, Start 2012)
 \item Webmoney
 \item Wirecard
\end{itemize}
 

%------------------------------------------------------------------------------------

\subsubsection{Wettbewerberdaten, sofern verfügbar und innderdeutsch relevant}

paysafecard.com Wertkarten AG   \newline \newline
Jahresumsatz 20010 i.H.v. 35 Millionen EUR 
20 Prozent Umsatz durch Sportwetten, hauptsächlich Glücksspiel 
Umsatz i.d. letzten Jahren regelmäßig 126 Prozent des Vorjahres \cite{d1} \newline
\newline Zielgruppe: Onlinenutzer, die bisher noch nicht im Internet eingekauft haben, z.
Zt. ca. 20 Millionen Menschen \newline \newline
Positionierung: Erste Prepaid-Karte für das Bezahlen im Internet, das
Unternehmen beansprucht Themenführerschaft in diesem Bereich \cite{d2} \newline
\newline
Paypal  \newline \newline
Umsatz 2010 i.H.v. 3,4 Milliarden USD \cite{d3}, \cite{d4}\newline
Umsatzziel 2013 i.H.v 6-7 Milliarden USD \cite{d4}\newline 
Wachstumsmarktsegment: Mobile Zahlungen \cite{d5}\newline \newline
Visa, Mastercard, Amex, Discover \newline \newline
Umsätze alleinstehend unbekannt, Zahlungsvariante ist Bestandteil der 
Produktpalette von Aggregatoren \newline \newline
Giropay\newline \newline
Betrieben von Postbank, Star Finanz und GAD (Volksbanken und Raiffeisenbanken)
und Fiducia IT, keine Finanzdaten verfügbar.\newline \newline
%------------------------------------------------------------------------------------

\subsection{Markt- und Wachstumscharakteristika}

Eine Studie der VRL Finance und Welsh Assembly \cite{da2} prognostiziert ein
Marktwachstum für Micropayment von jährlich 18 Prozent. Bis zum Jahr 2015 hat 
der innereuropäische Mircopayment-Markt ein Volumen von 15 Milliarden EUR,
ausgehend von 6 Milliarden EUR im Jahr 2010. \newline 

Einer Studie von Harris Interactive zufolge \cite{da3} nutzen 40 Prozent der
Internetnutzer in Deutschland Micropayment-Dienste 9,1 Mal pro Monat. Hierbei werden
Micropayment-Zahlungen als Zahlungen mit einem Wert kleiner als 10 EUR
definiert. Durchschnittlich werden pro Nutzer 19,30 EUR pro Monat über
Micropayment umgesetzt. \newline 

Im Jahr 2010 sind bei einem Marktvolumen von 6 Milliarden EUR somit in Europa
2,59 Millionen Micropayment-Nutzer vorhanden gewesen. \cite{da2}, \cite{da3} \newline 

Die 4 bekanntesten Browsergameanbieter, welche mittlerweile inoffiziellen
Sekundärquellen zufolge nahezu 100 Prozent Ihres Umsatzes aus dem
Micropayment-Bereich generieren, machten im Jahr 2010 zusammen ca 30 Millionen
EUR Gewinn, siehe Abschnitt \"Browsergame-Anbieter, innerdeutsch\". Genaue
Umsatzzahlen sind hier leider nicht verfügbar. \newline

Da der Zielmarkt der Geschäftsidee momentan derjenige der Onlinespieleanbieter
ist, sind die Browsergameanbieter primäres Ziel der Kundenakquise. \newline

Bei break even Umsatz von 200K EUR pro Jahr müssen 0,67 Prozent des Gewinns
dieser Browsergameanbieter als Umsatz für die angestrebte Unternehmung
abgeschöpft werden.  \newline 

200K EUR entsprechen bei 19,30 EUR Monatsumsatz pro User rein rechnerisch einem
Kundenstamm von unter 1000 Nutzern. \newline 

%------------------------------------------------------------------------------------

\subsection{Browsergame-Anbieter, innerdeutsch}

%------------------------------------------------------------------------------------

Gameforge AG, Karlsruhe\newline \newline
Geschäftsbericht 2009 weist Bilanzgewinn i.H.v. 26,3 Millionen EUR aus,
Wachstum um 246 Prozent im Vergleich zum Vorjahr \newline

Blue Byte GmbH\newline \newline
Jahresbilanz (bis 31.03.2010) weist Gewinn i.H.v 208K EUR aus, Wachstum um 86
Prozent im Vergleich zum Vorjahr  (BlueByte als Micropayment-Sparte von
Ubisoft)\newline 

InnoGames GmbH, Hamburg-Harburg\newline \newline
Bilanzgewinn 2009 i.H.v. ca 3 Millionen EUR im Vergleich zu 163K EUR im Jahr
2008\newline 

Bigpoint GmbH, Hamburg\newline \newline
Bilanzgewinn 2009 i.H.v. ca 564K EUR, ist aber komplett Gewinnvortrag aus dem
Vorjahr\newline
  
% vim: fileencoding=utf8
